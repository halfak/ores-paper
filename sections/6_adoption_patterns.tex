When we designed and developed ORES, we were targeting a specific problem: expanding the set values applied to the design of quality control tools to include recent a recent understanding of the importance of newcomer socialization.  We do not have any direct control of how developers chose to use ORES.  We hypothesize that, by making edit quality predictions available to all developers, we would lower the barrier to experimentation in this space.  From our experiences, it is clear that we lowered barriers to experimentation.  After we deployed ORES, we implemented some basic tools to showcase ORES, but we observed a steady adoption of our various prediction models by external developers in current tools and through the development of new tools.\footnote{See complete list: \url{http://enwp.org/:mw:ORES/Applications}}

\subsection{Adoption in current tools}
Many tools for counter-vandalism in Wikipedia were already available when we developed ORES.  Some of them made use of machine prediction (e.g. Huggle\footnote{Notably, Huggle adopted ORES prediction models soon after we deployed}, STiki, ClueBot NG), but most did not.  Soon after we deployed ORES, many developers that had not previously included their own prediction models in their tools were quick to adopt ORES.  For example, RealTime Recent Changes\footnote{\url{http://enwp.org/:m:RTRC}} includes ORES predictions along-side their realtime interface and FastButtons,\footnote{\url{http://enwp.org/:pt:Wikipédia:Scripts/FastButtons}} a Portuguese Wikipedia gadget, began displaying ORES predictions next to their buttons for quick reviewing and reverting damaging edits.

Other tools that were not targeted at counter-vandalism also found ORES predictions---specificly that of \emph{article quality} (wp10)---useful.  For example, RATER,\footnote{\url{http://enwp.org/::en:WP:RATER}} a gadget for supporting the assessment of article quality began to include ORES predictions to help their users assess the quality of articles and SuggestBot,\footnote{\url{http://enwp.org/User:SuggestBot}}\cite{cosley2007suggestbot} a robot for suggesting articles to an editor, began including ORES predictions in their tables of recommendations.

\subsection{Development of new tools}
\label{sec:new_tools}
Many new tools have been developed since ORES was released that may not have been developed at all otherwise.  For example, the Wikimedia Foundation product department developed a complete redesign on MediaWiki's Special:RecentChanges interface that implements a set of powerful filters and highlighting.  They took the ORES Review Tool to it's logical conclusion with an initiative that they referred to as Edit Review Filters.\footnote{\url{http://enwp.org/:mw:Edit_Review_Improvements}}  In this interface, ORES scores are prominently featured at the top of the list of available features, and they have been highlighted as one of the main benefits of the new interface to the editing community.

When we first developed ORES, English Wikipedia was the only wiki that we are aware of that had a robot that used machine prediction to automatically revert obvious vandalism\cite{carter2008cluebot}.  After we deployed ORES, several wikis developed bots of their own to use ORES predictions to automatically revert vandalism.  For example, PatruBOT in Spanish Wikipedia\footnote{\url{https://es.wikipedia.org/wiki/Usuario:PatruBOT}} and Dexbot in Persian Wikipedia\footnote{\url{https://fa.wikipedia.org/wiki/User:Dexbot}} now automatically revert edits that ORES predicts are damaging with high confidence.  These bots have been received with mixed acceptance.  Because of the lack of human oversight, concerns were raised about PatruBOT's false positive rate but after consulting with the developer, we were able to help them find an acceptable threshold of confidence for auto-reverts.

One of the most noteworthy new applications of ORES is the suite of tools developed by Sage Ross to support the Wiki Education Foundation's\footnote{\url{https://wikiedu.org/}} activities.  Their organization supports classroom activities that involve editing Wikipedia.  They develop tools and dashboards that help students contribute successfully and to help teachers monitor their students' work.  Ross has recently published about how he interprets meaning from ORES' article quality models\cite{ross2016visualizing} (an example of re-appropriation) and he has uses the article quality model in their new editor support dashboard\footnote{\url{https://dashboard-testing.wikiedu.org}} in a novel way to support new editors.  Specifically, Ross's tool\footnote{\url{https://dashboard-testing.wikiedu.org}} uses our feature injection system (see Section~\ref{sec:innovations_in_openness}) suggesting work to new editors.  This system asks ORES to score a student's draft and then asking ORES to reconsider the predicted quality level of the article with \emph{one more header}, \emph{one more image}, or \emph{one more citation}. In doing so, Ross built an intelligent user interface that can expose the internal structure of a model in order to recommend the most productive development to the article---the change that will most likely bring it to a higher quality level.
