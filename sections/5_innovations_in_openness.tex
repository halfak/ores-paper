Our goals in the development of ORES and the deployment of models is to keep the the flow of data, from random samples through model training, evaluation, and application, open for review, critique, and iteration.  In this section, we describe how we implemented transparent reproducibility in our model development process, and how ORES outputs a wealth of useful and nuanced information for users.  By making this detailed information available to users and developers, we hope to enable flexibility and power in the evaluation and use of ORES predictions for novel purposes.  In this section, we describe some of the key, novel innovations that have made ORES fit Wikipedian concerns and be flexible to re-appropriation.  The appendix also contains information about ORES' detailed prediction output (section~\ref{sec:appendix.score_documents}), how users and tools can adjust their use to model fitness (section~\ref{sec:appendix.model_information}, and how the whole model development workflow is made inspectable and replicable (section~\ref{sec:appendix.explicit_pipelines}).

\subsection{Collaboratively labeled data}
There are two primary strategies for gathering labeled data for ORES' models: found traces and manual labels.

\leadin{Found traces} For many models, the MediaWiki platform records a rich set of digital traces that can be assumed to reflect a useful human judgement.  For example, in Wikipedia, it is very common that damaging edits will eventually be reverted and that good edits will not be reverted.  Thus the revert action (and remaining traces) can be used to assume that the reverted edit is damaging.  We have developed a re-usable script\footnote{see \emph{autolabel} in \url{https://github.com/wiki-ai/editquality}} that when given a sample of edits, will label the edits as ``reverted\_for\_damage'' or not based on a set of constraints: edit was reverted within 48 hours, the reverting editor was not the same person, and the edit was not restored by another editor.

However, this ``reverted\_for\_damage'' label is problematic in that many edits are reverted not because they are damaging but because they are involved in some content dispute.  Operationalizing quality by exclusively measuring what persists in Wikipedia reinforces Wikipedia's well-known systemic biases, which is a similar problem in using found crime data in predictive policing.  Also, the label does not differentiate damage that is a good-faith mistake from damage that is intentional vandalism.  So in the case of damage prediction models, we only make use of the ``reverted\_for\_damage'' label when manually labeled data is not available.  However, ORES makes it possible to compare predictions made from this dataset with those made from others, and incongruences between models can be quite generative in thinking about when, where, and why to use a particular approach.

Another case of found traces is a score of article quality on a standardized, seven-point ordinal scale, which editors routinely make as part of a well-established process\footnote{\url{http://enwp.org/WP:ASSESS}}, named ``wp10'' after the Wikipedia 1.0 assessment project, where the article quality assessment scale originated\footnote{\url{http://enwp.org/WP:WP10}}.  We follow the process developed by Warncke-Wang et al.\cite{wang2017english} to extract the revision of an article that was current at the time of an assessment.  Many other wikis employ a similar process of article quality labeling (e.g. French Wikipedia and Russian Wikipedia), so we can use the same script to extract their assessments with some localization.\footnote{see \emph{extract\_labelings} in \url{https://github.com/wiki-ai/articlequality}}  However, other wikis either do not apply the same labeling scheme consistently (or at all), leaving manual labeling is our only option.

%\input{figures/wikilabels_screenshot}
\leadin{Manual labeling campaigns with Wiki Labels}
We hold manual labeling by human Wikipedians as the gold standard for purposes of training a model to replicate human judgement.  By asking Wikipedians to demonstrate their judgement on examples from their own wikis, we can most closely tailor model predictions to match the judgements that make sense to these communities.  This contrasts with found data, which deceptively appears to be a better option because of its apparent completeness: every edit was either reverted or not. Yet as previously discussed, there are many issues with bias, and the implicit signals may not exactly match the intended use of the model.  Manual labeling has a high up-front expense of human labor.  In order to efficiently utilize valuable time investments by our collaborators -- mostly volunteer Wikipedians -- we developed a system called ``Wiki Labels\footnote{\url{http://enwp.org/:m:Wiki labels}}''.  Wiki Labels allows Wikipedians to submit judgments of specific random samples of Wiki content using a convenient interface and logging in via their Wikipedia account.

For example, to supplement our models of edit quality, we replace the models based on found ``reverted\_for\_damage'' traces with manual judgments where we specifically ask labelers to distinguish ``damaging''/good from ``good-faith''/vandalism. ``Good faith'' is a well-established term in Wikipedian culture, with specific local meanings that are different than their broader colloquial use --- similar to how Wikipedians define ``consensus'' or ``neutrality''.  Using these labels we can build two separate models which allow users to filter for edits that are likely to be good-faith mistakes\cite{halfaker2017automated}, to just focus on vandalism, or to apply themselves broadly to all damaging edits.

\subsection{Threshold optimization}
When we first started developing ORES, we realized that operational concerns of Wikipedia's curators need to be translated into confidence thresholds for the prediction models.  For example, counter-vandalism patrollers seek to catch all (or almost all) vandalism before it stays in Wikipedia for very long.  That means they have an operational concern around the \emph{recall} of a damage prediction model.  They also like to review as few edits as possible in order to catch that vandalism.  So they have an operational concern around the \emph{filter rate}---the proportion of edits that are not flagged for review by the model\cite{halfaker2016notes}.

By finding the threshold of prediction likelihood that optimizes the filter-rate at a high level of recall, we can provide vandal-fighters with an effective trade-off for supporting their work.  We refer to these optimizations in ORES as \emph{threshold optimizations} and ORES provides information about these thresholds in a machine-readable format so that tools can automatically detect the relevant thresholds for their wiki/model context.

Originally, when we developed ORES, we defined these threshold optimizations in our deployment configuration.  But eventually, it became apparent that our users wanted to be able to search through fitness metrics to choose thresholds that matched their own operational concerns.  Adding new optimizations and redeploying quickly became a burden on us and a delay for our users.  In response, we developed a syntax for requesting an optimization from ORES in realtime using fitness statistics from the models tests. E.g. \texttt{maximum recall @ precision >= 0.9} gets a useful threshold for a counter-vandalism bot or \texttt{maximum filter\_rate @ recall >= 0.75} gets a useful threshold for semi-automated edit review (with human judgement).

\begin{figure}[htbp]
        \makebox{\hrulefill}{
        \small
        \begin{verbatim}
  {"threshold": 0.30, ...,
   "filter_rate": 0.88, "fpr": 0.097, "precision": 0.21, "recall": 0.75}
        \end{verbatim}
        \hrule
        \normalsize}
        \caption{Result of \url{https://ores.wikimedia.org/v3/scores/enwiki/?models=damaging&model_info=statistics.thresholds.true.'maximum filter_rate @ recall >= 0.75'}}
        \label{fig:english_damaging_threshold_optimization}
\end{figure}

This result shows that, when a threshold is set on 0.299 likelihood of damaging=true, then you can expect to get a recall of 0.751, precision of 0.215, and a filter-rate of 0.88.  While the precision is low, this threshold reduces the overall workload of vandal-fighters by 88\% while still catching 75\% of (the most egregious) damaging edits.

\subsection{Dependency injection and interrogability}
From a technical perspective, ORES is a algorithmic ``scorer'' container system.  It is primarily designed for hosting machine classifiers, but it is suited to other scoring paradigms as well.  For example, at one point, our experimental installation of ORES hosted a Flesch-Kincaid readability scorer.  The only real constraints are that the scorer must express its inputs in terms of ``dependencies'' that ORES knows how to solve and the scores must be presentable in JSON.

\subsubsection{Dependency injection}
One of the key features of ORES that allows scores to be generated in an efficient and flexible way is a dependency injection framework.

\leadin{Efficiency}
For example, there are several features that go into the ``damaging'' prediction model that are drawn from the \emph{diff} of two versions of an articles text (words added, words removed, badwords added, etc.)  Because all of these features ``depend'' on the \emph{diff}, the dependency solver can make sure that only one \emph{diff} is generated and that all subsequent features make use of that shared data.

ORES can serve multiple scores in the same request.  E.g. a user might want to gather a set of scores: ``edit type,'' ``damaging,'' and ``good-faith.''  Again, all of these prediction models depend on features related to the edit \emph{diff}.  ORES can safely combine all of the features required for each of the requested models and extract them with their shared dependencies together.  Given that feature extraction tends to take the majority of time in a real-time request (partially for fetching data [IO], partially for computation time [CPU]), this allows the scoring for multiple, similar models to take roughly the same amount of time as scoring a single model.

\leadin{Flexibility}
When developing a model, it's common to experiment with different feature sets.  A natural progression in the life of a model in the wild involves the slow addition of new features that prove useful during experimentation.  By implementing a dependency system and a dependency solver, a model can communicate to ORES which features it needs and ORES can provide those features.  At no point does a model developer need to teach ORES how to gather new data.  All of the information needed for the solver to solve any given feature set is wrapped up in the dependencies.

\subsubsection{Interrogability}
The flexibility provided by the dependency injection framework lets us implement a novel strategy for exploring \emph{how} ORES' models make predictions.  By exposing the features extracted to ORES users and allowing them to inject their own features, we can allow users to ask how predictions would change if the world were different.  Let's say you wanted to explore how ORES judges unregistered (anon) editors differently from registered editors.  Figure~\ref{fig:anon_injection} demonstrates two prediction requests to ORES.

\begin{figure}[h]
\centering
\begin{subfigure}[t]{.5\textwidth}
  \makebox{\hrulefill}{
  \small
  \begin{verbatim}
  "damaging": {
    "score": {
      "prediction": false,
      "probability": {
        "false": 0.938910157824447,
        "true": 0.06108984217555305
      }
    }
  }
  \end{verbatim}
  \hrule
  \normalsize}
  \caption{Prediction with \texttt{anon = false} injected}
  \label{fig:anon_injection_false}
\end{subfigure}~~
\begin{subfigure}[t]{.5\textwidth}
  \makebox{\hrulefill}{
  \small
  \begin{verbatim}
  "damaging": {
    "score": {
      "prediction": false,
      "probability": {
        "false": 0.9124151990561908,
        "true": 0.0875848009438092
      }
    }
  }
  \end{verbatim}
  \hrule
  \normalsize}
  \caption{Prediction with \texttt{anon = true} injected}
  \label{fig:anon_injection_true}
\end{subfigure}
\caption{Two ``damaging'' predictions about the same edit are listed for ORES.  In one case, ORES is asked to make a prediction assuming the editor is unregistered (anon) and in the other, ORES is asked to assume the editor is registered.}
\label{fig:anon_injection}
\end{figure}

Figure~\ref{fig:anon_injection_false} shows that ORES' ``damaging'' model concludes that the edit identified by the \emph{revision ID} of 34234210 is not damaging with 93.9\% confidence.  We can ask ORES to make a prediction about the exact same edit, but to assume that the editor was unregistered (anon). Figure~\ref{fig:anon_injection_true} shows the prediction if edit were saved by an anonymous editor.  ORES would still conclude that the edit was not damaging, but with less confidence (91.2\%).  By following a pattern like this for a single edit or a set of edits, we can get to know how ORES prediction models account for anonymity.

Interrogability is useful for checking to see where ORES's biases originate and what effects they have on predictions, as well as for creative new uses beyond bias.  For example, Ross uses our article quality prediction models (\emph{wp10}) as a method for suggesting work to new editors.\footnote{\url{https://dashboard-testing.wikiedu.org}}  This system asks ORES to score a student's draft and then asking ORES to reconsider the predicted quality level of the article with \emph{one more header}, \emph{one more image}, or \emph{one more citation}. In doing so, he built an intelligent user interface that can expose the internal structure of a model in order to recommend the most productive development to the article---the change that will most likely bring it to a higher quality level.
